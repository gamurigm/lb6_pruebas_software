\documentclass[12pt]{article}
\usepackage[utf8]{inputenc}
\usepackage[spanish]{babel}
\usepackage{graphicx}
\usepackage{listings}
\usepackage{xcolor}
\usepackage{geometry}
\usepackage{hyperref}

\geometry{a4paper, margin=2.5cm}

\definecolor{codegreen}{rgb}{0,0.6,0}
\definecolor{codegray}{rgb}{0.5,0.5,0.5}
\definecolor{codepurple}{rgb}{0.58,0,0.82}
\definecolor{backcolour}{rgb}{0.95,0.95,0.92}

\lstdefinestyle{mystyle}{
    backgroundcolor=\color{backcolour},   
    commentstyle=\color{codegreen},
    keywordstyle=\color{magenta},
    numberstyle=\tiny\color{codegray},
    stringstyle=\color{codepurple},
    basicstyle=\ttfamily\footnotesize,
    breakatwhitespace=false,         
    breaklines=true,                 
    captionpos=b,                    
    keepspaces=true,                 
    numbers=left,                    
    numbersep=5pt,                  
    showspaces=false,                
    showstringspaces=false,
    showtabs=false,                  
    tabsize=2
}

\lstset{style=mystyle}

\title{
    \includegraphics[width=0.4\textwidth]{logo_espe.png} \\ [1cm]
    \textbf{Laboratorio 6: CI/CD usando GitHub Actions} \\
    \large Universidad de las Fuerzas Armadas ESPE
}
\author{Estudiante: [Nombre del Estudiante] \\ Instructor: Ing. Enrique Calvopiña E, Mgtr.}
\date{\today}

\begin{document}

\maketitle
\newpage

\section{Introducción}
Este documento detalla el proceso de configuración de un flujo de Integración Continua (CI) y Entrega Continua (CD) utilizando GitHub Actions. Se implementan pruebas unitarias con Jest y análisis estático con ESLint sobre un servidor Express básico.

\section{Objetivos}
\begin{itemize}
    \item Configurar un flujo de CI en GitHub Actions para automatizar pruebas y linting.
    \item Implementar pruebas unitarias utilizando Jest.
    \item Aplicar análisis estático de código con ESLint.
\end{itemize}

\section{Desarrollo}

\subsection{Parte 1: Establecimiento de la estructura del proyecto base}
Se creó la estructura del proyecto incluyendo archivos de configuración y dependencias. Se utilizó el puerto 3111 para el servidor.

\begin{center}
    \textit{[Espacio para captura de pantalla: Estructura de archivos]}
    \vspace{5cm}
\end{center}

\subsection{Parte 2: Creación de archivos base}
A continuación se presenta el código fuente desarrollado:

\subsubsection{package.json}
\begin{lstlisting}[language=json]
// Consultar archivo package.json generado
\end{lstlisting}

\subsubsection{index.js}
\begin{lstlisting}[language=javascript]
import express from 'express';
const app = express();
const port = 3111;
// ... resto del codigo
\end{lstlisting}

\subsection{Parte 3: Configuración de Git y CI/CD}
Esta sección cubre los 4 pasos fundamentales solicitados:

\textbf{Paso 1: Crear repositorio en la cuenta de Git.}
\begin{itemize}
    \item a. Abrir la cuenta de Git en el navegador.
    \item b. Crear un nuevo repositorio vacío.
\end{itemize}

\textbf{Paso 2: Ejecución de comandos para clonar al repositorio.}
\begin{itemize}
    \item a. \texttt{git init}
    \item b. \texttt{git add .}
    \item c. \texttt{git commit -m "Proyecto base con CI"}
    \item d. \texttt{git branch -M main}
    \item e. \texttt{git remote add origin https://github.com/USUARIO/REPO.git}
    \item f. \texttt{git push -u origin main}
\end{itemize}

\textbf{Paso 3: Crear el workflow de GitHub Actions.}
\begin{itemize}
    \item a. Crear el archivo \texttt{.github/workflows/ci.yml}.
    \item b. Configurar disparadores (push, pull\_request).
    \item c. Definir trabajos (jobs) y pasos (checkout, install, lint, test).
\end{itemize}

\textbf{Paso 4: Probar la CI.}
\begin{itemize}
    \item a. Realizar cambios en el código.
    \item b. Ejecutar push.
    \item c. Verificar en la pestaña "Actions" de GitHub.
\end{itemize}

\section{Resultados y Evidencias}
En esta sección se deben incluir capturas de pantalla del flujo de GitHub Actions ejecutándose correctamente.

\begin{center}
    \textit{[Espacio para captura de pantalla: Workflow aprobado en GitHub]}
    \vspace{7cm}
\end{center}

\section{Conclusiones}
\begin{enumerate}
    \item GitHub Actions facilita la integración de pruebas automatizadas en el flujo de desarrollo.
    \item El uso de linting y pruebas unitarias garantiza un código más limpio y libre de errores básicos antes del despliegue.
\end{enumerate}

\section{Recomendaciones}
\begin{enumerate}
    \item Se recomienda configurar umbrales de cobertura de pruebas para asegurar la calidad del software.
    \item Utilizar secretos de GitHub para manejar información sensible en los flujos de trabajo.
\end{enumerate}

\end{document}

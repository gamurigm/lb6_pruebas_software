\section{Desarrollo}

\subsection{Parte 1: Establecimiento de la estructura del proyecto base}
Se creó la estructura del proyecto incluyendo archivos de configuración y dependencias. Se utilizó el puerto 3111 para el servidor.

\begin{figure}[h]
    \centering
    \includegraphics[width=0.7\textwidth]{img/cap0.png}
    \caption{Estructura inicial del proyecto}
    \label{fig:cap0}
\end{figure}

\subsection{Parte 2: Creación de archivos base}
A continuación se presenta el código fuente desarrollado:

\subsubsection{package.json}
\begin{lstlisting}[language=json]
{
  "name": "lab6-ci-cd",
  "version": "1.0.0",
  "main": "index.js",
  "type": "module",
  "scripts": {
    "start": "node index.js",
    "test": "node --experimental-vm-modules node_modules/jest/bin/jest.js",
    "lint": "eslint ."
  }
}
\end{lstlisting}

\subsubsection{index.js}
\begin{lstlisting}[language=javascript]
import express from 'express';
const app = express();
const port = 3111;

app.get('/', (req, res) => {
  res.send('CI/CD Lab - GitHub Actions');
});

app.listen(port, () => {
  console.log(`Server running at http://localhost:${port}`);
});
\end{lstlisting}

\subsection{Parte 3: Configuración de Git y CI/CD}
Esta sección cubre los 4 pasos fundamentales solicitados:

\textbf{Paso 1: Crear repositorio en la cuenta de Git.}
\begin{itemize}
    \item a. Abrir la cuenta de Git en el navegador.
    \item b. Crear un nuevo repositorio vacío.
\end{itemize}

\textbf{Paso 2: Ejecución de comandos para clonar al repositorio.}
\begin{itemize}
    \item a. \texttt{git init}
    \item b. \texttt{git add .}
    \item c. \texttt{git commit -m "Proyecto base con CI"}
    \item d. \texttt{git branch -M main}
    \item e. \texttt{git remote add origin https://github.com/USUARIO/REPO.git}
    \item f. \texttt{git push -u origin main}
\end{itemize}

\textbf{Paso 3: Crear el workflow de GitHub Actions.}
\begin{itemize}
    \item a. Crear el archivo \texttt{.github/workflows/ci.yml}.
    \item b. Configurar disparadores (push, pull\_request).
    \item c. Definir trabajos (jobs) y pasos (checkout, install, lint, test).
\end{itemize}

\begin{figure}[h]
    \centering
    \includegraphics[width=0.85\textwidth]{img/cap1-linter-tests.png}
    \caption{Ejecución de linter y tests en el pipeline CI/CD}
    \label{fig:cap1}
\end{figure}

\begin{figure}[h]
    \centering
    \includegraphics[width=0.85\textwidth]{img/cap2-server-running.png}
    \caption{Servidor ejecutándose correctamente}
    \label{fig:cap2}
\end{figure}

\textbf{Paso 4: Probar la CI.}
\begin{itemize}
    \item a. Realizar cambios en el código.
    \item b. Ejecutar push.
    \item c. Verificar en la pestaña "Actions" de GitHub.
\end{itemize}

\begin{figure}[h]
    \centering
    \includegraphics[width=0.9\textwidth]{img/parte4-paso4.1.png}
    \caption{Verificación del workflow en GitHub Actions - Vista general}
    \label{fig:parte4-paso4-1}
\end{figure}

\begin{figure}[h]
    \centering
    \includegraphics[width=0.9\textwidth]{img/parte4-paso4.2.png}
    \caption{Detalle de la ejecución del workflow en GitHub Actions}
    \label{fig:parte4-paso4-2}
\end{figure}
